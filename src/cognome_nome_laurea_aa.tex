% Template per Elaborato di Laurea
% DISI - Dipartimento di Ingegneria e Scienza dell’Informazione

% formato FRONTE RETRO
\documentclass[epsfig,a4paper,11pt,titlepage,twoside,openany]{book}
\usepackage{epsfig}
\usepackage{plain}
\usepackage{setspace}
\usepackage[paperheight=29.7cm,paperwidth=21cm,outer=1.5cm,inner=2.5cm,top=2cm,bottom=2cm]{geometry} % per definizione layout
\usepackage{titlesec} % per formato custom dei titoli dei capitoli

%%%%%%%%%%%%%%
% supporto lettere accentate
%\usepackage[latin1]{inputenc} % per Windows;
\usepackage[utf8x]{inputenc} % per Linux (richiede il pacchetto unicode);
%\usepackage[applemac]{inputenc} % per Mac.

\singlespacing

\usepackage[english]{babel}

\begin{document}

  % nessuna numerazione
  \pagenumbering{gobble} 
  \input{./chapters/first/pagina_iniziale}

  \clearpage
 
  \input{./chapters/ringraziamenti/ringraziamenti}
  \clearpage
  \pagestyle{plain} % nessuna intestazione e pie pagina con numero al centro

  
  % inizio numerazione pagine in numeri arabi
  \mainmatter



    % indice
    \tableofcontents
    \clearpage
    
    
          
    % gruppo per definizone di successione capitoli senza interruzione di pagina
    \begingroup

      \renewcommand{\cleardoublepage}{} 
      %\renewcommand{\clearpage}{} 
    % commentato altrimenti si rompe !!!!!!!!!!!!!!!!!!!!!!!!
      %\titleformat{\chapter}
       % {\normalfont\Huge\bfseries}{\thechapter}{1em}{}
        
      \titlespacing*{\chapter}{0pt}{0.59in}{0.02in}
      \titlespacing*{\section}{0pt}{0.20in}{0.02in}
      \titlespacing*{\subsection}{0pt}{0.10in}{0.02in}
      
      % sommario
      \chapter*{Abstract} % senza numerazione
\label{sommario}
\addcontentsline{toc}{chapter}{Abstract}
% 
We conducted an analysis on the reverts, i.e., when the version of a page is edited or restored to
that of a specific date, made on all the pages and users of Wikipedia in the Italian, Spanish,
Catalan and English versions. We detected specific patterns of reverts such as revert chains: a
series of reverts made by two or more users that continuously delete the previous revert on pages
regarding controversial topics or people. Mutual reverts are instead a pattern that happens when
inside a page two users are both reverting and being reverted by each other. Our goal is to improve
community health by focusing on experienced users. We designed and used state of the art metrics to
detect controversiality based on the revert pattern previously defined. A large portion of the work
was dedicated to the computation of different datasets from the Wikimedia history dumps: having many
small datasets is useful to gather the related data together and to have a faster computation of
them. The datasets generated can be divided into two different modules: chains, containing data
concerning revert chains, and groups, containing data about the different groups the involved users
belong to: admin, registered and anonymous. The last one is useful to classify users as experienced
or not. The generated datasets can be divided into two distinct categories; in the first one, the data
has been saved aggregated by pages, while in the second one by users. The former helps us to have an
idea about those pages in which most of the conflicts takes place and to which topics are related.
The latter is useful to understand how the different groups' activity is distributed on Wikipedia.
The data is saved month by month and this allows us to study the different metrics over time. We
have done a comparative analysis between languages to detect the differences between them. We have
discovered that the percentage of Italian users who have joined a chain is more than three times
higher than in the Catalan and English cases. In the analysis of the admins' influence, it emerged
how in Catalan one half of the reverts are made from admin towards registered users, while in
the Spanish one this value is way smaller. This is also confirmed by the fact that in the Spanish
Wikipedia there are 11 admins per million users versus the 56 in the Catalan one. From the analysis
of the reverts both done and received, we have found an anomaly in the Italian Wikipedia: the
reverts received by anonymous users are in a comparable number with the reverts received by the
registered ones, while in the other Wikipedias we observed how users receive very few reverts. The
analysis of the most controversial topics is also useful to detect sociopolitical issues and track
their development over time. 


\clearpage
\chapter*{Sommario} 
\addcontentsline{toc}{chapter}{Sommario}

Abbiamo condotto un'analisi sui reverts, cioè quando la versione di una pagina viene modificata o
ripristinata a quella di una data specifica, su tutte le pagine e gli utenti di Wikipedia nelle
versioni italiana, spagnola, catalano e inglese. Abbiamo individuato specifici patterns di reverts
come le catene di revert: una serie di revert fatti da due o più utenti che cancellano continuamente
il revert precedente su pagine riguardanti argomenti o persone controverse. I mutual revert sono
invece un pattern che si verifica quando all'interno di una pagina due utenti fanno e subiscono un
revert l'uno con l'altro. Il nostro obiettivo è quello di migliorare la community health
concentrandosi sugli utenti esperti. Abbiamo progettato e utilizzato metriche allo stato dell'arte
per rilevare la controversialità sulla base sui pattern di revert precedentemente definiti. Una gran
parte del lavoro è stata dedicata al calcolo di diversi dataset dal Wikimedia history dumps: avere
molti piccoli dataset è utile per raccogliere i dati tra loro correlati e per avere una velocità di
analisi maggiore. I dataset generati possono essere divisi in due diversi moduli: chains, contenenti
dati riguardo alle catene di revert, e groups, che contengono i dati relativi ai diversi gruppi a
cui appartengono gli utenti coinvolti: admin, registrati e anonimi. Quest'ultimo è utile per
classificare gli utenti come esperti o meno. I dataset generati possono essere divisi in due
categorie distinte; nella prima, i dati sono aggregati per pagina, mentre nel secondo per utente. Il
primo ci aiuta a capire in quali pagine avvengono la maggior parte dei conflitti e a quali argomenti
si riferiscono. Il secondo è utile per capire come è distribuita l'attività dei diversi gruppi di
utenti su Wikipedia. I dati vengono salvati mese per mese e questo ci permette di studiare le
diverse metriche nel tempo. Abbiamo fatto un'analisi comparata tra le lingue per rilevare le
differenze tra di esse. Abbiamo scoperto che la percentuale di utenti italiani che sono in una
catena è più di tre volte più alta che nei casi del catalano e dell'inglese. Nell'analisi
dell'influenza degli amministratori, è emerso come in catalano la metà dei reverts siano fatti dagli
amministratori nei confronti di utenti registrati, mentre in quello spagnolo questo valore è molto più
piccolo. Questo è confermato anche dal fatto che nella Wikipedia spagnola ci siano 11 admin per
milione di utenti contro i 56 di quella catalana. Dall'analisi dei reverts, sia fatti che ricevuti,
abbiamo trovato un'anomalia nella Wikipedia italiana: i revert ricevuti dagli utenti anonimi sono
in numero comparabile con i reverts ricevuti da quelli registrati, mentre nelle altre Wikipedie abbiamo
osservato come gli utenti registrati ricevano pochissimi reverts. L'analisi degli argomenti più
controversi è utile anche per individuare questioni sociopolitiche e tracciare il loro sviluppo nel
tempo. 

% We conducted an analysis on the
% reverts, i.e., when the version of a page is edited or restored to
% that of a specific date, made on all the pages of Wikipedia in Italian, Spanish, Catalan and on part
% of the English one. with the main focus on revert wars, a series of reverts made by two or more
% users that continuously delete the previous revert on pages regarding controversial topics or
% people. By computing datasets containing pages about controversialities, identifed by a metric based
% on the number of reverts, we studied both the revert chains and the groups the users belonged to,
% with the intent to explore the possibility of improving the community health. 

% The works could be devied in two moments, the actual analysis of the datsets and the generations of
% them. Several dataset had been computed from the wikimedia history dumps in order to have small datasets to speed un the
% analysis.The datasets computed could be devided in two different modules: one with the focus on
% revert chains and the other in which we wanted do divide the users in category: admin registered and
% anonymous. each dataset has been computed both by pages and users. 

% Thanks to these data processed we had reach some conclusions, especially doing a comparated analysis
% between languages.

% the data aggregated by pages allow us to understand which are the main controversial topics. the
% data is also saved month by month and this is useful to draw the trend of a page over the time 

% the data aggregated by users helps to identify the volumes of categories of users that perform and
% receive reverts. o to reconstruct the revert history of a user 

% experienced usersok
% community health ok
% controversiality ok
% pattern ok
% mutual reverts ok
% chains ok
% compute datasets 
% revert wars
% conflicts 

      % lista dei capitoli
      \clearpage
      \chapter{Introduction}
Wikipedia (WP) is the biggest source of information you can find on the internet, there are more
than 6 million articles and they are all maintained by volunteers. The value of Wikipedia is all in
the hands of the editors.

Many articles mean many users and therefore many potential conflicts. Avoiding these conflicts is
the best way for this encyclopedia to develop. \\ 
Each Wikipedia page has four different sections:  
\begin{itemize}
    \item Article: the actual content of the page.
    \item Talk Page: a forum where people can talk about edits. 
    \item History: a place where everyone can see the older versions of the pages.
    \item Source: in this section users can edit the page. 
\end{itemize}

Conflicts could happen both on the Talk page, through a discussion, and in the Article, through an
edit war. It is important to analyze all these aspects 

\begin{figure}
    \centering
    \includegraphics[width=1\textwidth]{./chapters/01/assets/wikipedia_page.png}
    \caption{page structure}
    \label{fig:page}
\end{figure}

\section{Main Project}
\label{sec:Main Project}
The project our team is working on, in collaboration between Eurecat and the Wikimedia Foundation,
is named: "Community Health Metrics: Understanding Editor Drop-off", this is a part of the
project idea : \\

“The primary value of Wikipedia is the editors. When an editor leaves the project, we lose their
participation and contribution to the community, This could be related to multiple factors, also
external to the project, but it could signal an issue related to internal dynamics and to the health
of the community. While a big effort was dedicated to retain new editors, we lack knowledge and
initiatives focused on understanding and preventing drop-off for experienced editors.”
\\

As stated in the description of the project, the focus is on the experienced users, which are the
core of Wikipedia: there are 41,741,926 Wikipedia accounts but the active users are only 132,916
namely
%\footnote{https://stats.wikimedia.org/EN/TablesWikipediansEditsGt5.htm}  %se metto footnote saltano le citazioni 
~ 3\% of all users. 

Focusing on this category of users and understanding the reasons that lead to a drop-off can give a
big help to WP. Several people are working on this project, this work is just a part of the whole.
In the team, everyone is working on a specific topic with the idea of then merging the different
results to obtain an analysis of the phenomenon from different points of view in order to have a
greater understanding of the life cycle of users.  

The prevention of the drop-off is not the only goal of the project, improving the community health
is also important to let users be in a good environment without being held back from editing. 

\section{My Contribution}
\label{sec:project}
The topic explored in this study is the revert analysis - i.e., when the version of a page is
restored to that of a specific date - for all the articles of Wikipedia.

This project consisted of the analysis of the edit history of different language editions of
Wikipedia to study patterns of reverts and edit wars to understand their potential effect on
individual user activity.

I implemented state-of-the-art metrics of controversy based on reverts and mutual reverts and
developed a new metric based on revert chains. Metrics have been computed per page and per user
monthly.

The results can be viewed in an interactive dashboard avaiable online.
\section{Releated Work}
There are different works that involves reverts: An interesting tool that allows to visualize the
conflicts~\cite{Suh2007} the problem is that it is from 2007 and wikipedia started growing around
2010 now that there are new technologies and much more data to analyze we can reach more interesting
conclusions. There have been analysis on antisocial behavior caused by vandalism~\cite{Kiesel2017}
but since the focus of the project is on experienced users this is not relevant for this study.  

      \chapter{Background}
Everyone knows what is Wikipedia and how to read an article, but there are many features
that most people are not aware of, \textit{e.g.} see all the versions of a page and being able to edit it. 
Anyone with a browser and without much effort can see and compare all the edits in a
Wikipedia page. For developers, there are many powerful resources such as big datasets
containing a lot more information.  

\section{History Exploration}
In the history section of a wikipedia article is possibile to see every version of the page.
There are several tools anyone can use to explore the revision history: 
% questa va se vuoi foto e testo vicini
% \begin{wrapfigure}[12]{r}{0.5\textwidth}[h]
%     \centering
%     \includegraphics[width=0.35\textwidth]{./chapters/02/assets/mobile_history.jpg}
%     \caption{mobile interactive visualization of the history}
%     \label{fig:mobilehistory}
% \end{wrapfigure}


\begin{itemize}
    \item Mobile application: this resource is only available on mobile device and provides us
        some statistics about the edits of the page like the total number of revisions (Fig \ref{fig:mobilehistory}). 
    \item Website: it is possible to compare two versions with an interactive tool that shows
        the progress of the modified page: each change corresponds to a bar indicating the number of
        bytes added or removed from the revision (Fig. \ref{fig:history}).
\end{itemize}

\begin{figure}[H]
    \centering
    \includegraphics[width=0.35\textwidth]{./chapters/02/assets/mobile_history.jpg}
    \caption{Mobile interactive visualization of the history.}
    \label{fig:mobilehistory}
\end{figure}

\begin{figure}[H]
    \centering
    \includegraphics[width=1\textwidth]{./chapters/02/assets/history.png}
    \caption{Interactive visualization of the history.}
    \label{fig:history}
\end{figure}


\section{Dataset}
There are two datasets that store information about Wikipedia edits made available from the WikiMedia
Foundation: a) the MediaWiki History\footnote{\url{https://wikitech.wikimedia.org/wiki/Analytics/Data_Lake/Edits/MediaWiki_history}} 
and b) the MediaWiki History Dumps \footnote{\url{https://wikitech.wikimedia.org/wiki/Analytics/Data_Lake/Edits/Mediawiki_history_dumps}}


The only difference other than the format (XML the former, TSV the latter) is that the former has the
page content. The dataset used in this study is the MediaWiki History Dumps.\\

Each line of the TSV represent an event and, since it is denormalized, the events for user, page and
revision are stored in the same schema.
All event entity have different event types:
\begin{itemize}
    \item Page: create, delete, move, reatore, merge  
    \item User: create, rename, altergroup (change user rights), alterblocks (block user)
    \item Revision: create (edit a page)
\end{itemize}

In this analysis only revision events are of interest, there are 68 fields but only a few were
needed. The entry could be divided in different sections: one section with general information of
the revision like timestamp and comment, a section with information about the user who did the
revision, one for the page where the revision was made, and the last one with more specific
information about the revision. The most relevant fields of each section are represented in the Tables 
\ref{table:user}, \ref{table:page}, \ref{table:revision}.The caption include, if
needed, the descriptions of the fields.

\begin{table}[H]
    \centering
    \ra{1.2}
    \begin{tabularx}{\columnwidth}{@{}Xccccc@{}}
        \midrule
        \textbf{id} & \textbf{username} & \textbf{groups} & \textbf{is\_anonymous} & \textbf{registration} & \textbf{revision\_count}\\ \toprule
        42081 & Checco & autopatrolled & False & 2006-02-10 14:52:44.0 & 10420 \\

         \bottomrule
    \end{tabularx}
    
    \caption{Data about the user who did the revision. The \textit{groups} field helps to
     identify if the user is an admin. The \textit{revision\_count} is needed to calculate complex metrics like M and G. \label{table:user}}
\end{table}


\begin{table}[H]
    \centering
    \ra{1.2}
    \begin{tabularx}{\columnwidth}{@{}Xccc@{}}
        \midrule
        \textbf{id} & \textbf{title} & \textbf{namespace} & \textbf{revision\_count} \\ \toprule
        116530 & Pino\_Rauti & 0 &  195 
        \\

         \bottomrule
    \end{tabularx}
    
    \caption{Data about about the page where the revision took place. The \textit{namespace} field is used
    to filter only the revisions from the namespace 0, i.e., the actual encyclopedia. \label{table:page}}
\end{table}

\begin{table}[H]
    \centering
    \ra{1.2}
    \begin{tabularx}{\columnwidth}{@{}Xcccc@{}}
        \midrule
        \textbf{id} & \textbf{parent\_id} & \textbf{is\_reverted} & \textbf{reverter\_id} & \textbf{is\_reverter} \\ \toprule
        73507165 & 73506955 & True &  73511400 & False 
        \\

         \bottomrule
    \end{tabularx}
    
    \caption{Data about the revision itself, we are able to identify if the revision is reverting another
    one, if it is been reverted and who is the reverter. \label{table:revision} }
\end{table}


\begin{table}[H]
    \centering
    \ra{1.2}
    \begin{tabularx}{\columnwidth}{@{}Xc@{}}
        \midrule
        \textbf{language} & \textbf{size}\\ \toprule
        English & 540 GB \\
        Spanish & 72 GB \\
        Italian & 54 GB \\
        Catalan & 12 GB \\

         \bottomrule
    \end{tabularx}
    
    \caption{Size of the dataset in different languages. \label{table:datasetsize}}
\end{table}

\section{Definitions}
It is worth defining some terms that will be used several times in the discussion.
\newtheorem{Definition}{Definition}
\begin{Definition}
    (Revert) On Wikipedia, reverting means undoing or otherwise negating the effects of one or more edits,
    which results in the page (or a part of it) being restored to a previous version. % TODO: citare il sito 
\end{Definition}

\begin{Definition}
    (Revert chain) On a Wikipedia page, a revert chain occurs when an edit that reverts an edit is itself reverted.
\end{Definition}

\begin{Definition}
    (Mutual revert) A “mutual revert” is recognized if a pair of editors (x, y) is observed once with x and once with y as the reverter~\cite{Yasseri2014}.
\end{Definition}

\begin{Definition}
    (Editor weight) The weight of an editor x is defined as the number of edits N performed by him or her~\cite{Yasseri2014}.
\end{Definition}

\begin{Definition}
    (Mutual revert weight) The weight of a mutually reverting pair MW is defined as the minimum of the weights of the two editors~\cite{Yasseri2014}.
\end{Definition}

\begin{Definition}
    (Chain weight) The weight of a revert chain CW is defined as the minimum of the weights of the editors involved in the chain.
\end{Definition}


\section{Metrics}
Two complex controversiality metrics have been computed in this study: the first one, M, is a state of the
art metric introduced by Yasseri \textit{et al.}~\cite{Yasseri2014} which give us a score of the controversiality of the page
based on the presence mutual reverts. The second one that we designed, called G, is very similar to M, but
instead of using mutual reverts, it uses revert chains to evaluate the controversiality of the page.  

\paragraph*{Controversiality M}
The controversiality M of an article is defined by summing the weights of all mutually reverting
editor pairs, excluding the topmost pair, and multiplying this number by the total number of editors
E involved in the article.

\begin{equation}
    M = E   \sum_{all\ mutual\ reverts} MW
\end{equation}

\paragraph*{Controversiality G}
The controversiality G of an article is defined by summing the weights of all the chains
there are on a page and multiplying by the total number of editors N involved in at least one chain.

\begin{equation}
    G = N \sum_{all\ revert\ chains} CW
\end{equation}





      \chapter{Methods}

Considering the huge dimension of the dataset and the fact that a large portion of its content was
useless, smaller datasets have been computed with the aim of expediting the analysis even for future
usages. The analysis was made based on the computed datasets. These datasets can be computed for
every language thanks to a bash script, in this way a multilingual analysis on the most
controversial topics can be conducted in different locations.


\bigskip



\tikzstyle{square} = [rectangle, rounded corners, minimum width=3cm, minimum height=1cm,text centered,text width=3cm, draw=black, fill=blue!20]
\tikzstyle{arrow} = [thick,->,>=stealth]

\begin{tikzpicture}[node distance=4cm]
    \node (dataset) [square, xshift=4cm] {MediaWiki History Dumps};
    \node (computed) [square, right of=dataset, xshift=2cm] {Computed Datasets};
    \node (anal) [square, right of=computed, xshift=2cm] {Analysis};

    \draw [arrow] (dataset) --node[anchor=south] {compute}(computed);
    \draw [arrow] (computed) -- node[anchor=south]{analyze}(anal);

\end{tikzpicture}

\bigskip

\section{Computed Dataset}
After the first skimming, only the revisions involving a revert were left. This dataset, whose
schema is the same as the MediaWiki History Dumps, has been sorted by both page and timestamp,the
size is now \~10\% of the originaland thanks to this screening. To achieve this we went through the
compressed dataset line by line decompressing it on the fly and saving in a file only the entries we
were interested in. Due to this the amount of ram required is very small and so is the disk space
since all data is compressed. For the sorting part, we used Unix sort, which is the most optimized
way to sort a file like this.  

From this filtered dataset have been computed several smaller datasets which can be divided into two modules: 

\begin{itemize}
    \item Chains: in these datasets, the focus was on detecting revert chains in pages 
    \item Group: in these datasets, the focus was on the number of reverts that users did or received based on the groups ( admin, registered, anonymous).
\end{itemize}

\subsection{Chains}
The data about revert chains were computed from the compressed filtered dataset. Every time we used
the filtered dataset we read it line by line, saving only the interesting information. The output
is a JSON file, where each page corresponds a JSON object. For each page we save the list of chains
and some statistics. A chain has a start and an end date, a list of revisions, and the users
involved. This dataset is way smaller than the initial one so it is possible to browse the dataset
in few seconds.\\


To identify a chain we used a function called \textit{simple\_chains} that differs from another called
\textit{complex\_chains} for the fact that the first one to identify a chain of revert considers only
contiguous reverts. We decided to use the simple one because we are only interested in chains that
occur in a short time span since that is where most of the discussions take place. If more than
50\% of users involved in a chain are bots the chain is not saved. There are two versions of this
dataset, one that considers anonymous users and one that does not. \\

In the schema below there are all the fields in a page object. 
\begin{verbatim}
    {
        "title": "Loligo_vulgaris", 
        "chains": 
        [{
            "revisions": ["113715375", "113715381", "113715393"], 
            "users": {"62.18.117.244": "", "Leo0428": "17181"}, 
            "len": 3, 
            "start": "2020-06-15 22:16:23.0", 
            "end": "2020-06-15 22:17:38.0"
        }], 
        "n_chains": 1, 
        "n_reverts_in_chains": 3, 
        "n_reverts": 38
        "mean": 3.0, 
        "longest": 3, 
        "G": 0,
        "M": 0, 
        "lengths": {"3": 1}
    }
    
\end{verbatim}


Regarding users, the object is very similar, but it is calculated differently. All the data we need
is stored in the JSON pages. By analyzing that file you can extract all the chains in which a user
has been involved and then calculate statistics in a similar way as for pages. Using this dataset
allow us to compute this dataset 10 times faster.  

The only difference is that the M field is missing because it is only related to a page. The field G, instead,
can be computed on a user considering every chain where it is the author of at least a revision.\\

The dataset was also calculated monthly for both users and pages, the schema is simpler than the
JSON and this allows us to save it in a TSV using only one row for each month. Instead of saving
all the data about the chain, we save the number of chains that are longer than 5,7,9. In Table
\ref{table:chainsPagemonth} there is a sample page entry. To do this we have processed the
JSON database one page (or user) at a time by dividing by month. We counted the chains per month basing
on the start date of the chain.   

\begin{table}[H]
    \centering
    \ra{1.2}
    \begin{tabularx}{\columnwidth}{@{}Xcccccccccc@{}}
        \midrule
        \textbf{title} & \textbf{year\_month} & \textbf{nchain} & \textbf{nrev} & \textbf{mean} & \textbf{longest} & \textbf{$\geq$ 5} & \textbf{$\geq$ 7} & \textbf{$\geq$ 9} & \textbf{G}\\ \toprule
        Loligo\_vulgaris & 2020-10 & 1 & 15 & 3.0 & 3 & 0 & 0 & 0 & 0\\
        
        \bottomrule
    \end{tabularx}
    
    \caption{entry of the mothly tsv \label{table:chainsPagemonth}}
\end{table}


\subsection{Group}
Another interesting part of the study was focusing on the category a user belongs. Thanks to this we
are able to track the habits of the users allowing us to understand, for example, if someone stops
editing Wikipedia after several reverts from admins. Detecting these kinds of patterns is useful for
community health: a user can be warned if its behavior could lead to a drop-off. The groups to which
users can belong are: 


\begin{itemize}
    \item Admin (sysop): can perform certain actions like blocking users and editing protected pages, 
    \item Registered: are logged in at the time of the edit, 
    \item Anonymous: are not logged in and their username is their IP address(it is not possible to match an IP with a user
        because the IP can change over time).
\end{itemize}

The datasets computed are both for pages and users: 
\paragraph*{Pages} 
For each page, there are two topics of investigation: reverts and mutual reverts. An entry of the
dataset is a page-month and gives us the number of reverts and mutual reverts made on the page
divided by group. This can be helpful, for example, to detect pages where admins are more active and
this could be a sign that something is wrong with the page.



The notation \textit{adm\_reg} in Table \ref{table:revertpage} refers to the number of admin that performed a
revert to a registered user (similarly with \textit{adm\_adm, reg\_adm, reg\_reg} ).\\

The notation \textit{mut\_ra} in the Table \ref{table:mutualpage} refers to the number of mutual reverts
where the users involved are a registered one and an admin, the order does not matter, in fact, there is
no \textit{mut\_ar} that would have the same value.\\


Since the focus was on experienced users, only pairs involving registered and admins were computed.
For having an idea of the volume of the reverts made by anon we saved the number of reverts
that were made by both anonymous (\textit{anon}) and not anonymous (\textit{not\_anon}).

To calculate these metrics we use simple variables that are incremented as you scroll through the filtered dataset
and initialized each time a new page is started. For both users and pages, we have discarded
edits that have been marked as vandalism and edits made by bots.

\begin{table}[H]
    \centering
    \ra{1.2}
    \begin{tabularx}{\columnwidth}{@{}Xcccccccccc@{}}
        \midrule
        \textbf{id} & \textbf{page} & \textbf{year\_month} & \textbf{adm\_adm} & \textbf{adm\_reg} & \textbf{reg\_adm} & \textbf{reg\_reg} & \textbf{anon} & \textbf{not\_anon}\\ \toprule
        1 & pagina & 2020-10 & 13 & 12 & 42 & 0 & 0 & 0 \\
        
         \bottomrule
    \end{tabularx}
    
    \caption{entry of the revert page tsv \label{table:revertpage}}
\end{table}

In the case of mutual reverts the procedure is similar but a bit more complex because we need to
store the information of the whole page in order to correctly detect all the mutual reverts. 
The most efficient way to save such information is the use of dictionaries where we saved for each
reverter the list of users who reverted and then at page processing time we computed a list of pairs
that were used to calculate the other metrics.

\begin{table}[H]
    \centering
    \ra{1.2}
    \begin{tabularx}{\columnwidth}{@{}Xcccccccccc@{}}
        \midrule
        \textbf{id} & \textbf{page} & \textbf{year\_month} & \textbf{mut\_aa} & \textbf{mut\_ra}  & \textbf{mut\_rr} & \textbf{anon} & \textbf{not\_anon}\\ \toprule
        1 & pagina & 2020-10 & 13 & 12 & 42  & 0 & 0 \\
         \bottomrule
    \end{tabularx}
    
    \caption{entry of the mutual page TSV \label{table:mutualpage}}
\end{table}

\paragraph*{User}
It is useful also to have the data aggregated by user. Reverts data can be retrieved from the
filtered dataset sorted by timestamp. The data about reverts is gathered and processed month by
month, this allowed us to save for each user-month the number of reverts made and received divided
by group.

In this case, the dataset is browsed and processed month by month. When a user performs a revert the
dataset gives us the id of the revision it has reverted but not the id of the user it has
reverted. To solve this problem so we had to save the info in different dictionaries: \textit{reverters, editor, groups}, 
\bigskip

reverters[username] gives us the list of the revision it reverted. \\
\indent editor[revision\_id] gives us the user who performs that edit. \\
\indent groups[username] gives us the groups a user belongs.\\


Combining this dictionaties we have all the data necessary to compute all the metrics we need.
\begin{table}[H]
    \centering
    \ra{1.2}
    \begin{tabularx}{\columnwidth}{@{}ccc@{}}
        \midrule
        \textbf{user} & \textbf{group} & \textbf{year\_month} \\ \toprule
        carlos & adm & 2020-10  \\
        
         \bottomrule
    \end{tabularx}
    \begin{tabularx}{\columnwidth}{@{}XXXXXXXX@{}}
        \midrule
        \textbf{received} & \textbf{r\_reg}  & \textbf{r\_not} & \textbf{r\_adm} & \textbf{done} & \textbf{d\_reg} & \textbf{d\_not} & \textbf{d\_adm}\\ \toprule
        13 & 12 & 42  & 0 & 13 & 12 & 42  & 0  \\
        
         \bottomrule
    \end{tabularx}
    
    \caption{entry of the mutual page tsv \label{table:revks}}
\end{table}


The mutual revert analysis was more difficult to implement because to save the information about
mutual reverts we need the dataset sorted by pages, but to get the data by user we should use the
one sorted by timestamp. We solved this problem by storing the user-page-month in the dataset, so
the information about the mutual returns of a user involved in a specific month on a specific page.
This led to a larger dataset but with a higher level of information: it is easy to post-process the dataset
by grouping by user or by month to have one entry per user or one entry per month,
respectively. 


\begin{table}[H]
    \centering
    \ra{1.2}
    \begin{tabularx}{\columnwidth}{@{}XXXXXXX@{}}
        \midrule
        \textbf{user} & \textbf{group} & \textbf{page\_name}& \textbf{year\_month} & \textbf{mut\_adm}& \textbf{mut\_reg}& \textbf{mut\_not}\\ \toprule
        khalu & adm & pagina & 2020-10 & 13 & 12 & 4 \\
        
         \bottomrule
    \end{tabularx}

    
    \caption{entry of the mutual page tsv \label{table:rjevks}}
\end{table}








      \chapter{Results and Discussion}

The second step of this work was the analysis of the dataset just generated. Thanks to the structure
and the heavy pruning analyzing these datasets is fast, This allows us to have a better workflow
without interruptions. We analyzed the data in two ways: a descriptive statistic and an interactive
one.
\paragraph*{Descriptive}
For each dataset there is a script that runs and plots various statistics using the python libraries
Pandas and Matplotlib. There are two types of output: plots and rankings. 
Plots are useful to understand the trend from a more comprehensive point of view month by month.  
Rankings instead are used to see in a more specific way the pages/users ordered by one of the
metrics previously computed. 
\paragraph*{Interactive}
We decided to make available online an interactive dashboard. The idea is
that everyone can change a few parameters and see how the metrics are performing in a personalized
way. To achieve this we uploaded our dataset on a database and thanks to an innovative way to retrieve
data (grapQL) we can display it on a website. 

\section{Chains}
From the analysis of the chains we can have an overview of an entire Wikipedia in a language,
discovering which is the mean lenght of chains or the longest chain. Another aspect worth
investigating is the relationship between alone reverts and reverts that are in a chain: more
reverts in chains means more discussions, in this cases we could combine the data of other team
members who analyzed the talk pages. While the pages chain are useful to have a less specific but
wider view of the phenomenon, the users chain let us see if a specific user is involved in many
chains and in which page is more active: in this sense we can define category of users: the ones who
are active just in some topic or the other who reverts an all wikipedia. More interesting are the
metrics by month, we can plot the trend of reverts in a page and see if it is always controversial
or just in a specific storic moment related to something happened in the world. Plotting the metrics
year by year allow us to understand the global activity of the users on wikipedia. Regarding users,
we can define the lifecycle of a users and see when is more active and combining the data with the
other team members we can say if its decrese of revisions  it is related to a discussion.

\subsection{Page}
Here the page ranked by the number of chains in italian and catalan, we can see that 6/10 were
football-related while in catalunya we can see, as expected, a stronger territorial belonging. It's
interesting how in catalan , for people, the second surname is written in the title of the article
but not in the spanish one
\begin{table}[H]
    \centering
    \ra{1.2}
    \begin{tabularx}{\columnwidth}{@{}Xcccc@{}}
        \midrule
        \textbf{id} & \textbf{title} & \textbf{n\_chains}& \textbf{title} & \textbf{n\_chains} \\ \toprule
        1 & Serie A & 195  & Barcelona & 68\\
        2 & Juventus FC & 190  & FC Barcelona & 33\\
        3 & Matteo Renzi & 179  & Catalunya & 30\\
        4 & AS Roma & 176  &País Valencià& 26\\
        5 & Personale della WWE & 167  &Marc Márquez i Alentà & 22\\
        6 & SSC Napoli & 162  & Mireia Belmonte i García& 22\\
        7 & Inter  & 162  &Girona & 20\\
        8 & Roma & 154 & Rafael Nadal i Parera & 19\\
        9 & Tiziano Ferro & 141 & Oriol Junqueras i Vies& 17\\
        10 & Gianluigi Buffon & 137  &Català & 16\\
        
         \bottomrule
    \end{tabularx}
    
    \caption{pages with more chains \label{table:chainspage}}
\end{table}

Let's see more specifically the italian first one, serie\_A, if we analyze the history we can see
that the biggest part of the chains, especially the longest, are caused by the number of championship won by Juventus \url{https://en.wikipedia.org/wiki/Calciopoli} 
\begin{verbatim}
    "title": Serie_A,
    "revisions": [...] 
    "n_chains": 195,
    "n_reverts_in_chains": 756,
    "n_reverts": 5291,
    "mean": 3.9,
    "longest": 15,
    "G": 2205218,
    "M": 9479660,
    "lunghezze": {"3": 96, "4": 66, "5": 15, "6": 11,"7": 2,"8": 3,"10": 1,"15": 1}
\end{verbatim}

pino rauti la piu lunga fottuto fascista di merda 

\paragraph*{monthly}
let's see the trend of month by month of this page 
\subsection{User}
\paragraph*{wars}
\paragraph*{monthly}



\section{Group}
From the analysis of the groups we can define different ranking of pages using the number of reverts
of each group, or given a page we can plot the trend of the edits by group and detect the pages in
which admins are more interested. It is possibile for each user to say if he is target of reverts
from or if it is an admin reverted and the ration between reverted made and received. You can do so
much analysis of this data, that is the reason why it is avaiable to everyone who needs it. 

\subsection{Page}
\paragraph*{reverts}
\paragraph*{mutual}
\subsection{User}
\paragraph*{reverts}
\paragraph*{mutual}







      \chapter{Infrastructure}

All the code is available on Github
\footnote{\url{https://github.com/WikiCommunityHealth/wikimedia-revert}} where there is an
organization called WikiCommunityHealth in which every team member gives its contribution to the
project. For the data processing, we used python since it's the best option to handle such an amount of
data. The data is currently stored in the Unitn servers of the Cricca group.
\paragraph*{Multi Language}
All the computed datasets are the results of several python scripts launched one by one. All the
work has been done using Italian Wikipedia as an example. Automatizing the process allows us to run
all the scripts in different languages without further effort. In order to achieve this automation we
used a bash script that takes the language as parameters e.g \textit{./generate\_dataset it} takes
the data from the Wikimedia History Dumps in Italian, then creates a folder called "it" and all the required subfolders
and then it generates the dataset in the right location. The only requirement is that the dump
must have already been downloaded. 


\section{Workflow}

\paragraph{File}
To compute the datasets, since we have to process huge amounts of data, we decided to use a
simple style of programming without using complex libraries. We used dictionaries as an example of
one of the most efficient data structures. Even if the computed dataset were different we always used
the same structure of the code, that consisted in a few steps, the program reads the compressed
Wikimedia History Dumps line by line and for each line :
\begin{enumerate}
    \item Parse from the dataset the pieces of information.
    \item Insert in dictionaries the information we want to save.
    \item Check if the page id is different from the previous one, if this is true it means that the
    page is finished so we can process it and initialize all the variables for a new page.
    \item If the page is not finished, we check if the month is finished and similarly to the page
    we process the information we gathered since we want to save an entry for each month.
    \item If neither the page nor the month is finished we can check if this revision is reverting the
    previous one or doing the computation we needed in that specific file.
\end{enumerate}

\section{Repository}
To handle a big project it is always important to organize in the best way possible the repository to avoid
confusion while browsing. Also the naming of the file is important.
\paragraph{Folder}
This is the structure of the folder where the code is organized, the bash folder contains all the used bash
scripts. The utility one has all the python files that were used to check various things, for example,
some files let us extract from the data about a specific pagefrom the dataset. The main
folder has all the files concerning the computation and analysis of the datasets, while the
database one has the script to upload the files on the database for the interactive dashboard.

\dirtree{%
.1 bash.
.1 main.
.2 analyze.
.2 create.
.2 database.
.1 utility.
}
\paragraph{Naming}
All the files in the main directory follow strict naming rules. The format is : \\
\texttt{type\_class\_aggregation\_name\_month\_format.py}
\begin{itemize}
    \item \textit{type} the file can be one that creates (c) or that analyses (a) a dataset
    \item \textit{class} since this work could be divided into 2 different sections, Chain and Group, we used
    them as the main identifier to classify the files. There is also a generic class used when the computed data
    wasn't neither a chain nor a group.
    \item \textit{aggregation} if the file is cathegorized by user or by page 
    \item \textit{name} the name of the metrics it computes
    \item \textit{month} (optional) if the file is cathegorized by month 
    \item \textit{format} In create files the format of the output file is written directly in the file
    name in order to quickly understand what kind of data it handles (TSV or JSON)
\end{itemize}

\section{Data}
A lot of different files were created in the process, so they must be well organized in order to retrieve them
without errors. There is a folder for each class, chains and groups, and for both of them there is
a folder for page and one for users.

\paragraph*{Folder structure}
Here is the folder structure of how the data was stored :  

\dirtree{%
.1 ita.
.2 chains.
.3 user.
.4 wars.json.
.4 monthly.tsv.
.3 page.
.4 wars.json.
.4 monthly.tsv.
.3 page\_reg.
.4 wars.json.
.2 group.
.3 user.
.4 mutuals.tsv.
.4 reverts.tsv.
.4 all.tsv.
.3 page.
.4 mutuals.tsv.
.4 reverts.tsv.
.4 all.tsv.
}

\paragraph*{Bash Script}
The main code is written in python but for some of the tasks we decided that using a bash
script was a better idea, in particular in order to automatize processes like downloading the Wikimedia History Dumps or
generating the datasets.






      \chapter{Conclusions}
This is still an open project, so the results described in this report are not complete. The biggest
part of the time has been dedicated to the computation of the datasets, for the analysis defined in the project
description we should combine the data of all project members. Nevertheless, with the generated
datasets it is possible to draw some conclusions. We have seen how the language, and so the place,
of the users, characterize Wikipedia. We have seen just some language but the study could be extended
to all available languages without problems.\\

Future works will comprehend, other than the study in different languages, an interactive dashboard
available online with this data and one of the other group members. this allows the users to
dynamically retrieve the data and plot the results as they wish.\\

Wikipedia is full of vandals but fortunately, they are neutralised fastly. From the number of chains we
can understand which are the topic in which people cannot reach an agreement; in Italy and south
America these topics are sport, especially football, in Catalunya the most debated topics are about
territorial belonging.



    \endgroup



    \addcontentsline{toc}{chapter}{Bibliografia}
    \bibliographystyle{plain}
    \bibliography{biblio}

    \titleformat{\chapter}
        {\normalfont\Huge\bfseries}{Allegato \thechapter}{1em}{}
    % sezione Allegati - opzionale
    \appendix
    \input{./chapters/allegati/allegati}

\end{document}


%%%%%%%%%%%%%%%%%%%%%%%%%%%%%%%%%%%%%%%%%%%%%%%%%%%%%%%%%%%%%%%%%%%%%%%%%%
%%%%%%%%%%%%%%%%%%%%%%%%%%%%%%%%%%%%%%%%%%%%%%%%%%%%%%%%%%%%%%%%%%%%%%%%%%
%% Nota
%%%%%%%%%%%%%%%%%%%%%%%%%%%%%%%%%%%%%%%%%%%%%%%%%%%%%%%%%%%%%%%%%%%%%%%%%%
%% Si ricorda che il numero massimo di facciate e' 30.
%% Nel conteggio delle facciate sono incluse 
%%   indice
%%   sommario
%%   capitoli
%% Dal conteggio delle facciate sono escluse
%%   frontespizio
%%   ringraziamenti
%%   allegati    
%%%%%%%%%%%%%%%%%%%%%%%%%%%%%%%%%%%%%%%%%%%%%%%%%%%%%%%%%%%%%%%%%%%%%%%%%%
%%%%%%%%%%%%%%%%%%%%%%%%%%%%%%%%%%%%%%%%%%%%%%%%%%%%%%%%%%%%%%%%%%%%%%%%%%
%%%%%%%%%%%%%%%%%%%%%%%%%%%%%%%%%%%%%%%%%%%%%%%%%%%%%%%%%%%%%%%%%%%%%%%%%%
%%%%%%%%%%%%%%%%%%%%%%%%%%%%%%%%%%%%%%%%%%%%%%%%%%%%%%%%%%%%%%%%%%%%%%%%%%
%% Nota
%%%%%%%%%%%%%%%%%%%%%%%%%%%%%%%%%%%%%%%%%%%%%%%%%%%%%%%%%%%%%%%%%%%%%%%%%%
%% Nella bibliografia devono essere riportati tutte le fonti consultate 
%% per lo svolgimento della tesi. La bibliografia deve essere redatta 
%% in ordine alfabetico sul cognome del primo autore. 
%% 
%% La forma della citazione bibliografica va inserita secondo la fonte utilizzata:
%% 
%% LIBRI
%% Cognome e iniziale del nome autore/autori, la data di edizione, titolo, casa editrice, eventuale numero dell’edizione. 
%% 
%% ARTICOLI DI RIVISTA
%% Cognome e iniziale del nome autore/autori, titolo articolo, titolo rivista, volume, numero, numero di pagine.
%% 
%% ARTICOLI DI CONFERENZA
%% Cognome e iniziale del nome autore/autori (anno), titolo articolo, titolo conferenza, luogo della conferenza (città e paese), date della conferenza, numero di pagine. 
%% 
%% SITOGRAFIA
%% La sitografia contiene un elenco di indirizzi Web consultati e disposti in ordine alfabetico. 
%% E’ necessario:
%%   Copiare la URL (l’indirizzo web) specifica della pagina consultata
%%   Se disponibile, indicare il cognome e nome dell’autore, il titolo ed eventuale sottotitolo del testo
%%   Se disponibile, inserire la data di ultima consultazione della risorsa (gg/mm/aaaa).    
%%%%%%%%%%%%%%%%%%%%%%%%%%%%%%%%%%%%%%%%%%%%%%%%%%%%%%%%%%%%%%%%%%%%%%%%%%
%%%%%%%%%%%%%%%%%%%%%%%%%%%%%%%%%%%%%%%%%%%%%%%%%%%%%%%%%%%%%%%%%%%%%%%%%%
    
