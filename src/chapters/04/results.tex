\chapter{Results and Discussion}

The second step of this work was the analysis of the dataset just generated. Thanks to the structure
and the heavy pruning analyzing these datasets is fast, This allows us to have a better workflow
without interruptions. We analyzed the data in two ways: a descriptive statistic and an interactive
one.
\paragraph*{Descriptive}
For each dataset there is a script that runs and plots various statistics using the python libraries
Pandas and Matplotlib. There are two types of output: plots and rankings. 
Plots are useful to understand the trend from a more comprehensive point of view month by month.  
Rankings instead are used to see in a more specific way the pages/users ordered by one of the
metrics previously computed. 
\paragraph*{Interactive}
We decided to make available online an interactive dashboard. The idea is
that everyone can change a few parameters and see how the metrics are performing in a personalized
way. To achieve this we uploaded our dataset on a database and thanks to an innovative way to retrieve
data (grapQL) we can display it on a website. 

\section{Chains}
From the analysis of the chains we can have an overview of an entire Wikipedia in a language,
discovering which is the mean lenght of chains or the longest chain. Another aspect worth
investigating is the relationship between alone reverts and reverts that are in a chain: more
reverts in chains means more discussions, in this cases we could combine the data of other team
members who analyzed the talk pages. While the pages chain are useful to have a less specific but
wider view of the phenomenon, the users chain let us see if a specific user is involved in many
chains and in which page is more active: in this sense we can define category of users: the ones who
are active just in some topic or the other who reverts an all wikipedia. More interesting are the
metrics by month, we can plot the trend of reverts in a page and see if it is always controversial
or just in a specific storic moment related to something happened in the world. Plotting the metrics
year by year allow us to understand the global activity of the users on wikipedia. Regarding users,
we can define the lifecycle of a users and see when is more active and combining the data with the
other team members we can say if its decrese of revisions  it is related to a discussion.

\subsection{Page}
Here the page ranked by the number of chains in italian and catalan, we can see that 6/10 were
football-related while in catalunya we can see, as expected, a stronger territorial belonging. It's
interesting how in catalan , for people, the second surname is written in the title of the article
but not in the spanish one
\begin{table}[H]
    \centering
    \ra{1.2}
    \begin{tabularx}{\columnwidth}{@{}Xcccc@{}}
        \midrule
        \textbf{id} & \textbf{title} & \textbf{n\_chains}& \textbf{title} & \textbf{n\_chains} \\ \toprule
        1 & Serie A & 195  & Barcelona & 68\\
        2 & Juventus FC & 190  & FC Barcelona & 33\\
        3 & Matteo Renzi & 179  & Catalunya & 30\\
        4 & AS Roma & 176  &País Valencià& 26\\
        5 & Personale della WWE & 167  &Marc Márquez i Alentà & 22\\
        6 & SSC Napoli & 162  & Mireia Belmonte i García& 22\\
        7 & Inter  & 162  &Girona & 20\\
        8 & Roma & 154 & Rafael Nadal i Parera & 19\\
        9 & Tiziano Ferro & 141 & Oriol Junqueras i Vies& 17\\
        10 & Gianluigi Buffon & 137  &Català & 16\\
        
         \bottomrule
    \end{tabularx}
    
    \caption{pages with more chains \label{table:chainspage}}
\end{table}

Let's see more specifically the italian first one, serie\_A, if we analyze the history we can see
that the biggest part of the chains, especially the longest, are caused by the number of championship won by Juventus \url{https://en.wikipedia.org/wiki/Calciopoli} 
\begin{verbatim}
    "title": Serie_A,
    "revisions": [...] 
    "n_chains": 195,
    "n_reverts_in_chains": 756,
    "n_reverts": 5291,
    "mean": 3.9,
    "longest": 15,
    "G": 2205218,
    "M": 9479660,
    "lunghezze": {"3": 96, "4": 66, "5": 15, "6": 11,"7": 2,"8": 3,"10": 1,"15": 1}
\end{verbatim}

pino rauti la piu lunga fottuto fascista di merda 

\paragraph*{monthly}
let's see the trend of month by month of this page 
\subsection{User}
\paragraph*{wars}
\paragraph*{monthly}



\section{Group}
From the analysis of the groups we can define different ranking of pages using the number of reverts
of each group, or given a page we can plot the trend of the edits by group and detect the pages in
which admins are more interested. It is possibile for each user to say if he is target of reverts
from or if it is an admin reverted and the ration between reverted made and received. You can do so
much analysis of this data, that is the reason why it is avaiable to everyone who needs it. 

\subsection{Page}
\paragraph*{reverts}
\paragraph*{mutual}
\subsection{User}
\paragraph*{reverts}
\paragraph*{mutual}






