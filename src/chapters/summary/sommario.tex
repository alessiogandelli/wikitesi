\chapter*{Abstract} % senza numerazione
\label{sommario}
\addcontentsline{toc}{chapter}{Abstract}
% 
We conducted an analysis on the reverts, i.e., when the version of a page is edited or restored to
that of a specific date, made on all the pages and users of Wikipedia in the Italian, Spanish,
Catalan and English versions. We detected specific patterns of reverts such as revert chains: a
series of reverts made by two or more users that continuously delete the previous revert on pages
regarding controversial topics or people. Mutual reverts are instead a pattern that happens when
inside a page two users are both reverting and being reverted by each other. Our goal is to improve
community health by focusing on experienced users. We designed and used state of the art metrics to
detect controversiality based on the revert pattern previously defined. A large portion of the work
was dedicated to the computation of different datasets from the Wikimedia history dumps: having many
small datasets is useful to gather the related data together and to have a faster computation of
them. The datasets generated can be divided into two different modules: chains, containing data
concerning revert chains, and groups, containing data about the different groups the involved users
belong to: admin, registered and anonymous. The last one is useful to classify users as experienced
or not. The generated datasets can be divided into two distinct categories; in the first one, the data
has been saved aggregated by pages, while in the second one by users. The former helps us to have an
idea about those pages in which most of the conflicts takes place and to which topics are related.
The latter is useful to understand how the different groups' activity is distributed on Wikipedia.
The data is saved month by month and this allows us to study the different metrics over time. We
have done a comparative analysis between languages to detect the differences between them. We have
discovered that the percentage of Italian users who have joined a chain is more than three times
higher than in the Catalan and English cases. In the analysis of the admins' influence, it emerged
how in Catalan one half of the reverts are made from admin towards registered users, while in
the Spanish one this value is way smaller. This is also confirmed by the fact that in the Spanish
Wikipedia there are 11 admins per million users versus the 56 in the Catalan one. From the analysis
of the reverts both done and received, we have found an anomaly in the Italian Wikipedia: the
reverts received by anonymous users are in a comparable number with the reverts received by the
registered ones, while in the other Wikipedias we observed how users receive very few reverts. The
analysis of the most controversial topics is also useful to detect sociopolitical issues and track
their development over time. 


\clearpage
\chapter*{Sommario} 
\addcontentsline{toc}{chapter}{Sommario}

Abbiamo condotto un'analisi sui reverts, cioè quando la versione di una pagina viene modificata o
ripristinata a quella di una data specifica, su tutte le pagine e gli utenti di Wikipedia nelle
versioni italiana, spagnola, catalano e inglese. Abbiamo individuato specifici patterns di reverts
come le catene di revert: una serie di revert fatti da due o più utenti che cancellano continuamente
il revert precedente su pagine riguardanti argomenti o persone controverse. I mutual revert sono
invece un pattern che si verifica quando all'interno di una pagina due utenti fanno e subiscono un
revert l'uno con l'altro. Il nostro obiettivo è quello di migliorare la community health
concentrandosi sugli utenti esperti. Abbiamo progettato e utilizzato metriche allo stato dell'arte
per rilevare la controversialità sulla base sui pattern di revert precedentemente definiti. Una gran
parte del lavoro è stata dedicata al calcolo di diversi dataset dal Wikimedia history dumps: avere
molti piccoli dataset è utile per raccogliere i dati tra loro correlati e per avere una velocità di
analisi maggiore. I dataset generati possono essere divisi in due diversi moduli: chains, contenenti
dati riguardo alle catene di revert, e groups, che contengono i dati relativi ai diversi gruppi a
cui appartengono gli utenti coinvolti: admin, registrati e anonimi. Quest'ultimo è utile per
classificare gli utenti come esperti o meno. I dataset generati possono essere divisi in due
categorie distinte; nella prima, i dati sono aggregati per pagina, mentre nel secondo per utente. Il
primo ci aiuta a capire in quali pagine avvengono la maggior parte dei conflitti e a quali argomenti
si riferiscono. Il secondo è utile per capire come è distribuita l'attività dei diversi gruppi di
utenti su Wikipedia. I dati vengono salvati mese per mese e questo ci permette di studiare le
diverse metriche nel tempo. Abbiamo fatto un'analisi comparata tra le lingue per rilevare le
differenze tra di esse. Abbiamo scoperto che la percentuale di utenti italiani che sono in una
catena è più di tre volte più alta che nei casi del catalano e dell'inglese. Nell'analisi
dell'influenza degli amministratori, è emerso come in catalano la metà dei reverts siano fatti dagli
amministratori nei confronti di utenti registrati, mentre in quello spagnolo questo valore è molto più
piccolo. Questo è confermato anche dal fatto che nella Wikipedia spagnola ci siano 11 admin per
milione di utenti contro i 56 di quella catalana. Dall'analisi dei reverts, sia fatti che ricevuti,
abbiamo trovato un'anomalia nella Wikipedia italiana: i revert ricevuti dagli utenti anonimi sono
in numero comparabile con i reverts ricevuti da quelli registrati, mentre nelle altre Wikipedie abbiamo
osservato come gli utenti registrati ricevano pochissimi reverts. L'analisi degli argomenti più
controversi è utile anche per individuare questioni sociopolitiche e tracciare il loro sviluppo nel
tempo. 

% We conducted an analysis on the
% reverts, i.e., when the version of a page is edited or restored to
% that of a specific date, made on all the pages of Wikipedia in Italian, Spanish, Catalan and on part
% of the English one. with the main focus on revert wars, a series of reverts made by two or more
% users that continuously delete the previous revert on pages regarding controversial topics or
% people. By computing datasets containing pages about controversialities, identifed by a metric based
% on the number of reverts, we studied both the revert chains and the groups the users belonged to,
% with the intent to explore the possibility of improving the community health. 

% The works could be devied in two moments, the actual analysis of the datsets and the generations of
% them. Several dataset had been computed from the wikimedia history dumps in order to have small datasets to speed un the
% analysis.The datasets computed could be devided in two different modules: one with the focus on
% revert chains and the other in which we wanted do divide the users in category: admin registered and
% anonymous. each dataset has been computed both by pages and users. 

% Thanks to these data processed we had reach some conclusions, especially doing a comparated analysis
% between languages.

% the data aggregated by pages allow us to understand which are the main controversial topics. the
% data is also saved month by month and this is useful to draw the trend of a page over the time 

% the data aggregated by users helps to identify the volumes of categories of users that perform and
% receive reverts. o to reconstruct the revert history of a user 

% experienced usersok
% community health ok
% controversiality ok
% pattern ok
% mutual reverts ok
% chains ok
% compute datasets 
% revert wars
% conflicts 