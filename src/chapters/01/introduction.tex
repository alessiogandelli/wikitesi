\chapter{Introduction}
Wikipedia (WP) is the biggest source of information you can find on the internet, there are more than 6 million articles and they are all
maintained by volunteers. The value of Wikipedia is all in the hands of the editors.

Many articles mean many users and therefore many potential conflicts. Avoiding these conflicts is the best way 
for this encyclopedia to develop. 
 \\ 
Each Wikipedia page has four different sections:  
\begin{itemize}
    \item Article: the actual content of the page.
    \item Talk Page: a forum where people can talk about edits. 
    \item History: a place where everyone can see the older versions of the pages.
    \item Source: in this section users can edit the page. 
\end{itemize}

Conflicts could happen both on the Talk page, through a discussion, and in the Article, through an edit war. It is important to analyze 
all these aspects 

\begin{figure}
    \centering
    \includegraphics[width=1\textwidth]{./chapters/01/assets/wikipedia_page.png}
    \caption{page structure}
    \label{fig:page}
\end{figure}

\section{Main Project}
\label{sec:Main Project}
The project in collaboration between Eurecat and the Wikimedia Foundation is named: "Community Health Metrics: Understanding Editor Drop-off" and this is 
a part of the project idea (?):
\\

“The primary value of Wikipedia is the editors. When an editor leaves the project, we lose their participation and contribution to the community, This could be related to multiple factors, also external to the project, but it could signal an issue related to internal dynamics and to the health of the community. While a big effort was dedicated to retain new editors, we lack knowledge and initiatives focused on understanding and preventing drop-off for experienced editors.”
\\

As stated in the description of the project, the focus is on the experienced users, which are the core of Wikipedia: there are 
41,741,926 Wikipedia accounts but the active users are only 132,916 namely  ~ 3\% of all users. 

Focusing on this category of users and understanding the reasons that lead to a drop-off can give a big help to WP. 
Several people are working on this project, this work is just a part of the whole.
In the team, everyone is working on a specific topic and the idea is to combine the different results to obtain an analysis of the
phenomenon from different points of view in order to have a greater understanding of the life cycle of users.  

The prevention of the drop-off is not the only goal of the project, improving the community health is also important to let users be
in a good environment without being held back from editing. 

\section{My Contribution}
\label{sec:project}
The topic explored in this study is the revert analysis - i.e., when the version of a page is restored to that of a specific date -
for all the articles of Wikipedia.


This project consisted of the analysis of the edit history of different language editions of Wikipedia to study patterns of reverts 
and edit wars to understand their potential effect on individual user activity.

I implemented state-of-the-art metrics of controversy based on reverts and mutual reverts and developed a new
metric based on revert chains. Metrics have been computed per page and per user monthly.
\section{Releated Work}
There are different works that involves reverts: 
An interesting tool that allows to visualize the conflicts 
\section{Context}
\label{sec:context}

\cite{pippo}




[1] https://stats.wikimedia.org/EN/TablesWikipediansEditsGt5.htm \\