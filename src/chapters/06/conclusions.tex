\chapter{Conclusions}
This is still an open project, so the results described in this report are not complete. The biggest
part of the time has been dedicated to the computation of the datasets, for the analysis defined in the project
description we should combine the data of all project members. Nevertheless, with the generated
datasets it is possible to draw some conclusions. We have seen how the language, and so the place,
of the users, characterize Wikipedia. We have seen just some language but the study could be extended
to all available languages without problems.\\

Future works will comprehend, other than the study in different languages, an interactive dashboard
available online with this data and one of the other group members. this allows the users to
dynamically retrieve the data and plot the results as they wish.\\

Wikipedia is full of vandals but fortunately, they are neutralised fastly. From the number of chains we
can understand which are the topic in which people cannot reach an agreement; in Italy and south
America these topics are sport, especially football, in Catalunya the most debated topics are about
territorial belonging.

