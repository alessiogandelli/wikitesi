\chapter{Conclusions}
This is still an open project, so the results described in this report are not complete. The largest
portion of the time has been dedicated to the computation of the datasets. For the analysis defined in the project
description we should combine the data of all project members. Nevertheless, with the generated
datasets it is possible to draw some conclusions. We have seen how the language, and so the living place,
of the users, characterizes Wikipedia. We have seen only some languages but the study could be extended
to all the available languages without any problem.\\

Future works will comprehend, besides the studies in different languages, an interactive dashboard
with this data available online and one from the other group members. This allows the users to
dynamically retrieve the data and plot the results as they wish.\\

Wikipedia is full of vandals but fortunately, they are quickly neutralised. From the number of chains we
can understand in which topic which people cannot reach an agreement; in Italy and south
America these topics are sport, especially football, while in Catalunya the most debated topics are about
territorial belonging.

