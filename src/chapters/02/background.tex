\chapter{Background}
Everyone knows what is Wikipedia and how to read an article, but there are many features
that most people are not aware of, \textit{e.g.} see all the versions of a page and being able to edit it. 
Anyone with a browser and without much effort can see and compare all the edits in a
Wikipedia page. For developers, there are many powerful resources such as big datasets
containing a lot more information.  

\section{History Exploration}
In the history section of a wikipedia article is possibile to see every version of the page.
There are several tools anyone can use to explore the revision history: 
% questa va se vuoi foto e testo vicini
% \begin{wrapfigure}[12]{r}{0.5\textwidth}[h]
%     \centering
%     \includegraphics[width=0.35\textwidth]{./chapters/02/assets/mobile_history.jpg}
%     \caption{mobile interactive visualization of the history}
%     \label{fig:mobilehistory}
% \end{wrapfigure}


\begin{itemize}
    \item Mobile application: this resource is only available on mobile device and provides us
        some statistics about the edits of the page like the total number of revisions (Fig \ref{fig:mobilehistory}). 
    \item Website: it is possible to compare two versions with an interactive tool that shows
        the progress of the modified page: each change corresponds to a bar indicating the number of
        bytes added or removed from the revision (Fig. \ref{fig:history}).
\end{itemize}

\begin{figure}[H]
    \centering
    \includegraphics[width=0.35\textwidth]{./chapters/02/assets/mobile_history.jpg}
    \caption{Mobile interactive visualization of the history.}
    \label{fig:mobilehistory}
\end{figure}

\begin{figure}[H]
    \centering
    \includegraphics[width=1\textwidth]{./chapters/02/assets/history.png}
    \caption{Interactive visualization of the history.}
    \label{fig:history}
\end{figure}


\section{Dataset}
There are two datasets that store information about Wikipedia edits made available from the WikiMedia
Foundation: a) the MediaWiki History\footnote{\url{https://wikitech.wikimedia.org/wiki/Analytics/Data_Lake/Edits/MediaWiki_history}} 
and b) the MediaWiki History Dumps \footnote{\url{https://wikitech.wikimedia.org/wiki/Analytics/Data_Lake/Edits/Mediawiki_history_dumps}}


The only difference other than the format (XML the former, TSV the latter) is that the former has the
page content. The dataset used in this study is the MediaWiki History Dumps.\\

Each line of the TSV represent an event and, since it is denormalized, the events for user, page and
revision are stored in the same schema.
All event entity have different event types:
\begin{itemize}
    \item Page: create, delete, move, reatore, merge  
    \item User: create, rename, altergroup (change user rights), alterblocks (block user)
    \item Revision: create (edit a page)
\end{itemize}

In this analysis only revision events are of interest, there are 68 fields but only a few were
needed. The entry could be divided in different sections: one section with general information of
the revision like timestamp and comment, a section with information about the user who did the
revision, one for the page where the revision was made, and the last one with more specific
information about the revision. The most relevant fields of each section are represented in the Tables 
\ref{table:user}, \ref{table:page}, \ref{table:revision}.The caption include, if
needed, the descriptions of the fields.

\begin{table}[H]
    \centering
    \ra{1.2}
    \begin{tabularx}{\columnwidth}{@{}Xccccc@{}}
        \midrule
        \textbf{id} & \textbf{username} & \textbf{groups} & \textbf{is\_anonymous} & \textbf{registration} & \textbf{revision\_count}\\ \toprule
        42081 & Checco & autopatrolled & False & 2006-02-10 14:52:44.0 & 10420 \\

         \bottomrule
    \end{tabularx}
    
    \caption{Data about the user who did the revision. The \textit{groups} field helps to
     identify if the user is an admin. The \textit{revision\_count} is needed to calculate complex metrics like M and G. \label{table:user}}
\end{table}


\begin{table}[H]
    \centering
    \ra{1.2}
    \begin{tabularx}{\columnwidth}{@{}Xccc@{}}
        \midrule
        \textbf{id} & \textbf{title} & \textbf{namespace} & \textbf{revision\_count} \\ \toprule
        116530 & Pino\_Rauti & 0 &  195 
        \\

         \bottomrule
    \end{tabularx}
    
    \caption{Data about about the page where the revision took place. The \textit{namespace} field is used
    to filter only the revisions from the namespace 0, i.e., the actual encyclopedia. \label{table:page}}
\end{table}

\begin{table}[H]
    \centering
    \ra{1.2}
    \begin{tabularx}{\columnwidth}{@{}Xcccc@{}}
        \midrule
        \textbf{id} & \textbf{parent\_id} & \textbf{is\_reverted} & \textbf{reverter\_id} & \textbf{is\_reverter} \\ \toprule
        73507165 & 73506955 & True &  73511400 & False 
        \\

         \bottomrule
    \end{tabularx}
    
    \caption{Data about the revision itself, we are able to identify if the revision is reverting another
    one, if it is been reverted and who is the reverter. \label{table:revision} }
\end{table}


\begin{table}[H]
    \centering
    \ra{1.2}
    \begin{tabularx}{\columnwidth}{@{}Xc@{}}
        \midrule
        \textbf{language} & \textbf{size}\\ \toprule
        English & 540 GB \\
        Spanish & 72 GB \\
        Italian & 54 GB \\
        Catalan & 12 GB \\

         \bottomrule
    \end{tabularx}
    
    \caption{Size of the dataset in different languages. \label{table:datasetsize}}
\end{table}

\section{Definitions}
It is worth defining some terms that will be used several times in the discussion.
\newtheorem{Definition}{Definition}
\begin{Definition}
    (Revert) On Wikipedia, reverting means undoing or otherwise negating the effects of one or more edits,
    which results in the page (or a part of it) being restored to a previous version. % TODO: citare il sito 
\end{Definition}

\begin{Definition}
    (Revert chain) On a Wikipedia page, a revert chain occurs when an edit that reverts an edit is itself reverted.
\end{Definition}

\begin{Definition}
    (Mutual revert) A “mutual revert” is recognized if a pair of editors (x, y) is observed once with x and once with y as the reverter~\cite{Yasseri2014}.
\end{Definition}

\begin{Definition}
    (Editor weight) The weight of an editor x is defined as the number of edits N performed by him or her~\cite{Yasseri2014}.
\end{Definition}

\begin{Definition}
    (Mutual revert weight) The weight of a mutually reverting pair MW is defined as the minimum of the weights of the two editors~\cite{Yasseri2014}.
\end{Definition}

\begin{Definition}
    (Chain weight) The weight of a revert chain CW is defined as the minimum of the weights of the editors involved in the chain.
\end{Definition}


\section{Metrics}
Two complex controversiality metrics have been computed in this study: the first one, M, is a state of the
art metric introduced by Yasseri \textit{et al.}~\cite{Yasseri2014} which give us a score of the controversiality of the page
based on the presence mutual reverts. The second one that we designed, called G, is very similar to M, but
instead of using mutual reverts, it uses revert chains to evaluate the controversiality of the page.  

\paragraph*{Controversiality M}
The controversiality M of an article is defined by summing the weights of all mutually reverting
editor pairs, excluding the topmost pair, and multiplying this number by the total number of editors
E involved in the article.

\begin{equation}
    M = E   \sum_{all\ mutual\ reverts} MW
\end{equation}

\paragraph*{Controversiality G}
The controversiality G of an article is defined by summing the weights of all the chains
there are on a page and multiplying by the total number of editors N involved in at least one chain.

\begin{equation}
    G = N \sum_{all\ revert\ chains} CW
\end{equation}




